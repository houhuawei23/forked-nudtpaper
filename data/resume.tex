\begin{resume}
\ifisanon
    \ifisblind
        \isanonfalse
    \else
        \ifismaster
        该论文作者在学期间取得的阶段性成果(学术论文等)已满足我校硕士学位评阅相关要求。为避免阶段性成果信息对专家评价学位论文本身造成干扰,特将论文作者的阶段性成果信息隐去。
        \else
        该论文作者在学期间取得的阶段性成果(学术论文等)已满足我校博士学位评阅相关要求。为避免阶段性成果信息对专家评价学位论文本身造成干扰,特将论文作者的阶段性成果信息隐去。
        \fi
    \fi
\else\relax
\fi

\ifisanon\relax
\else
\ifisresumebib

\begingroup
    \settoggle{bbx:gbtype}{false}%局部设置不输出文献类型和载体标识符
	\settoggle{bbx:gbannote}{true}%局部设置输出注释信息
    \renewcommand{\bibfont}{\normalsize}%\fangsong
	%\setcounter{gbnamefmtcase}{6}%局部设置作者的格式为familyahead格式
    \makeatletter
    \renewcommand*{\mkbibnamegiven}[1]{%通过作者注释局部调整作者的格式需与bib配合
    \ifitemannotation{thesisauthor}
    {\ifbibliography{{\textbf{#1}}}{#1}}%
    {#1}\ifbibliography{\ifitemannotation{corresponding}{\textsuperscript{*}}{}}{}%
    }
    \renewcommand*{\mkbibnamefamily}[1]{%
    \ifitemannotation{thesisauthor}
    {\ifbibliography{{\textbf{#1}}}{#1}}
    {#1}}
    \def\blx@maxbibnames{9}
    \def\blx@minbibnames{2}
    %\defcounter{gbbiblocalcase}{1} %局部强迫中文本地化字符串
    %\defcounter{gbbiblocalcase}{2} %局部强迫英文本地化字符串
    %\setlocalbibstring{andotherscn}{et al.}
    %\setlocalbibstring{andothers}{等}
    \renewbibmacro*{author}{%
      \ifboolexpr{
        test \ifuseauthor
        and
        not test {\ifnameundef{author}}
      }
        {\ifisblind%
           \setcounter{numitemval}{1}%
           \whileboolexpr{test {\ifnumcomp{\value{numitemval}}{<}{\value{author}+1}}}
               {\ifitemannotation[author][default][\thenumitemval]{thesisauthor}%
                    {\setcounter{numitemslt}{\thenumitemval}}{}%
               \stepcounter{numitemval}%
               }%
           \ifnumcomp{\value{numitemslt}}{>}{0}
                {\textbf{\iffieldequalstr{userd}{chinese}{第\zhnum{numitemslt}作者}{\Ordinalstring{numitemslt}~Author}}, XXX.}
                {\printnames{author}}
         \else%
            \printnames{author}%
         \fi%%
         \iffieldundef{authortype}
           {}
           {\setunit{\printdelim{authortypedelim}}%
            \usebibmacro{authorstrg}}}
        {}}
    \makeatother


	\begin{refsection}[ref/resume.bib]
	\nocite{ref-1-1-Yang,ref-2-1-杨轶,ref-3-1-杨轶,ref-4-1-Yang,
    ref-5-1-Wu,ref-6-1-贾泽,ref-7-1-伍晓明}
	
    \printbibliography[heading=subbibliography,title={发表的学术论文}]
	\end{refsection}

	\begin{refsection}[ref/resume.bib]
	\nocite{ref-8-1-任天令,ref-9-1-Ren}%%
	\printbibliography[heading=subbibliography,title={申请的发明专利}]
	\end{refsection}


	\begin{refsection}[ref/resume.bib]
	\nocite{比赛2021}%
	\printbibliography[heading=subbibliography,title={获得的竞赛奖项}]
	\end{refsection}

	\begin{refsection}[ref/resume.bib]
	\nocite{国自科2024}%
	\printbibliography[heading=subbibliography,title={参与的科研项目}]
	\end{refsection}
\endgroup


\else

  \section*{发表的学术论文} % 发表的和录用的合在一起

  \begin{enumerate}[label={[\arabic*]},labelsep=6pt,topsep=0pt,partopsep=0pt,
parsep=0pt,itemsep=1pt,leftmargin=0em,itemindent=42pt]
  \item Yang Y, Ren T L, Zhang L T, et al. Miniature microphone with silicon-
    based ferroelectric thin films. Integrated Ferroelectrics, 2003,
    52:229-235. (SCI 收录, 检索号:758FZ.)
  \item 杨轶, 张宁欣, 任天令, 等. 硅基铁电微声学器件中薄膜残余应力的研究. 中国机
    械工程, 2005, 16(14):1289-1291. (EI 收录, 检索号:0534931 2907.)
  \item 杨轶, 张宁欣, 任天令, 等. 集成铁电器件中的关键工艺研究. 仪器仪表学报,
    2003, 24(S4):192-193. (EI 源刊.)
  \item Yang Y, Ren T L, Zhu Y P, et al. PMUTs for handwriting recognition. In
    press. (已被 Integrated Ferroelectrics 录用. SCI 源刊.)
  \item Wu X M, Yang Y, Cai J, et al. Measurements of ferroelectric MEMS
    microphones. Integrated Ferroelectrics, 2005, 69:417-429. (SCI 收录, 检索号
    :896KM.)
  \item 贾泽, 杨轶, 陈兢, 等. 用于压电和电容微麦克风的体硅腐蚀相关研究. 压电与声
    光, 2006, 28(1):117-119. (EI 收录, 检索号:06129773469.)
  \item 伍晓明, 杨轶, 张宁欣, 等. 基于MEMS技术的集成铁电硅微麦克风. 中国集成电路,
    2003, 53:59-61.
  \end{enumerate}

  \section*{申请的发明专利} % 有就写,没有就删除
  \begin{enumerate}[label={[\arabic*]},labelsep=6pt,topsep=0pt,partopsep=0pt,
parsep=0pt,itemsep=1pt,leftmargin=0em,itemindent=42pt]
  \item 任天令, 杨轶, 朱一平, 等. 硅基铁电微声学传感器畴极化区域控制和电极连接的
    方法: 中国, CN1602118A. (中国专利公开号.)
  \item Ren T L, Yang Y, Zhu Y P, et al. Piezoelectric micro acoustic sensor
    based on ferroelectric materials: USA, No.11/215, 102. (美国发明专利申请号.)
  \end{enumerate}

  \section*{获得的竞赛奖项} %

\begin{enumerate}[label={[\arabic*]},labelsep=6pt,topsep=0pt,partopsep=0pt,
parsep=0pt,itemsep=1pt,leftmargin=0em,itemindent=42pt]
\normalsize
\item 2021 年XXX大赛, 国
家一等奖, 亚军.
\end{enumerate}

\section*{参与的科研项目} %

  \begin{enumerate}[label={[\arabic*]},labelsep=6pt,topsep=0pt,partopsep=0pt,
  parsep=0pt,itemsep=1pt,leftmargin=0em,itemindent=42pt]
  \normalsize
  \item XXX下的XXX方法研究(国家自然科学基金面上项目). %2024.01-2027.12
  \end{enumerate}

\fi
\fi
\end{resume}


