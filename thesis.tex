%!TEX program = xelatex
%!BIB program = biber
%%
%% This is file `thesis.tex',
%% generated with the docstrip utility.
%%
%% The original source files were:
%%
%% nudtpaper.dtx  (with options: `thesis')
%%
%% This is a generated file.
%%
%% Copyright (C) 2020 by Liu Benyuan <liubenyuan@gmail.com>
%%
%% This file may be distributed and/or modified under the
%% conditions of the LaTeX Project Public License, either version 1.3a
%% of this license or (at your option) any later version.
%% The latest version of this license is in:
%%
%% http://www.latex-project.org/lppl.txt
%%
%% and version 1.3a or later is part of all distributions of LaTeX
%% version 2004/10/01 or later.
%%
%% To produce the documentation run the original source files ending with `.dtx'
%% through LaTeX.
%%
%% Any Suggestions : LiuBenYuan <liubenyuan@gmail.com>
%% Thanks Xue Ruini <xueruini@gmail.com> for the thuthesis class!
%% Thanks sofoot for the original NUDT paper class!
%%
%1. 规范硕士导言
% \documentclass[master,ttf]{nudtpaper}
%2. 规范博士导言
% \documentclass[doctor,twoside,ttf]{nudtpaper}
%3. 建议使用OTF字体获得较好的页面显示效果
%   OTF字体从网上获得,各个系统名称统一。
%   如果你下载的是最新的(1201)OTF英文字体,建议修改nudtpaper.cls,使用
%   Times New Roman PS Std
% \documentclass[doctor,twoside,otf]{nudtpaper}
%   另外,新版的论文模板提供了方正字体选项FZ,效果也不错哦
% \documentclass[doctor,twoside,fz]{nudtpaper}
%4. 如果想生成盲评,传递anon即可,仍需修改个人成果部分
% \documentclass[master,otf,anon]{nudtpaper}
%
%5. 参考文献若用biblatex生成,则使用biber选项
% \documentclass[master,biber]{nudtpaper}
%
%6. 简历中的论文和成果用biblatex参考文献方式生成,则使用resumebib选项
% \documentclass[master,biber,resumebib]{nudtpaper}
%
%7. 如果是专硕,则使用prof选项
% \documentclass[master,biber,prof]{nudtpaper}
%
\documentclass[doctor,twoside,biber,resumebib,fz]{nudtpaper}
\addbibresource[location=local]{ref/refs.bib}
\usepackage{mynudt}

\classification{TP957}
\serialno{0123456}
\confidentiality{公开}
\UDC{}
\title{国防科大学位论文\LaTeX{}模板\\
使用手册}
\displaytitle{国防科技大学学位论文\LaTeX{}模板}
\author{张三}
\zhdate{\zhtoday}
\entitle{How to Use the \LaTeX{} Document Class for NUDT Dissertations}
\enauthor{ZHANG San}
\endate{\entoday}
\subject{通信与信息工程}
\ensubject{Information and Communication Engineering}
\researchfield{自动目标识别与模糊工程}
\supervisor{李四\quad{}教授}
\cosupervisor{王五\quad{}副教授} % 没有就空着
\ensupervisor{Prof. LI Si}
\encosupervisor{} % 没有就空着
\papertype{工学}
\enpapertype{Engineering}
% 加入makenomenclature命令可用nomencl制作符号列表。

\begin{document}
\graphicspath{{figures/}}
% 制作封面,生成目录,插入摘要,插入符号列表 \\
% 默认符号列表使用denotation.tex,如果要使用nomencl \\
% 需要注释掉denotation,并取消下面两个命令的注释。 \\
% cleardoublepage% \\
% printnomenclature% \\
\maketitle
\frontmatter

\let\cleardoublepage=\clearpage %章节前无需从单数页起始
\tableofcontents
\listoftables
\listoffigures

\midmatter
\begin{cabstract}
国防科学技术大学是一所直属中央军委的综合性大学。1984年,学校经国务院、中央军委和教育部批准首批成立研究生院,%
肩负着为全军培养高级科学和工程技术人才与指挥人才,培训高级领导干部,从事先进武器装备和国防关键技术研究的重要任务。%
国防科技大学是全国重点大学,也是全国首批进入国家“211工程” 建设并获中央专项经费支持的全国重点院校之一。%
学校前身是1953年创建于哈尔滨的中国人民解放军军事工程学院,简称“哈军工”。
\end{cabstract}
\ckeywords{国防科学技术大学;211;哈军工;间隔符用全角分号}

\begin{eabstract}
National University of Defense Technology is a comprehensive national key university based in Changsha, %
Hunan Province, China. It is under the dual supervision of the Ministry of National Defense %
and the Ministry of Education, designated for Project 211 and Project 985, %
the two national plans for facilitating the development of Chinese higher education. %

NUDT was originally founded in 1953 as the Military Academy of Engineering in Harbin of Heilongjiang Province. %
In 1970 the Academy of Engineering moved southwards to Changsha and was renamed Changsha Institute of Technology.%
 The Institute changed its name to National University of Defense Technology in 1978.

\end{eabstract}
\ekeywords{NUDT, MND, ME, Delimiter is comma}


\input{data/denotation}

%书写正文,可以根据需要增添章节; 正文还包括致谢,参考文献与成果
\let\cleardoublepage=\cleardoublepagewithhead %章节需从单数页起始
\mainmatter
\input{data/chap01}
\input{data/chap02}

\input{data/ack}

\cleardoublepage
\phantomsection
\addcontentsline{toc}{chapter}{参考文献}
\ifisbiber
{\hyphenpenalty=500 %
\tolerance=9900 %
\renewcommand{\baselinestretch}{1.35}
\printbibliography[heading=bibliography, title=参考文献]

}
\else
\bibliographystyle{bstutf8}
\bibliography{ref/refs}
\fi

\begin{resume}
\ifreview

\ifismaster
该论文作者在学期间取得的阶段性成果(学术论文等)已满足我校硕士学位评阅相关要求。为避免阶段性成果信息对专家评价学位论文本身造成干扰,特将论文作者的阶段性成果信息隐去。
\else
该论文作者在学期间取得的阶段性成果(学术论文等)已满足我校博士学位评阅相关要求。为避免阶段性成果信息对专家评价学位论文本身造成干扰,特将论文作者的阶段性成果信息隐去。
\fi

\else

\ifisresumebib

	\begin{refsection}[ref/resume.bib]
	\settoggle{bbx:gbtype}{false}%局部设置不输出文献类型和载体标识符
	\settoggle{bbx:gbannote}{true}%局部设置输出注释信息
	\setcounter{gbnamefmtcase}{1}%局部设置作者的格式为familyahead格式
	\nocite{ref-1-1-Yang,ref-2-1-杨轶,ref-3-1-杨轶,ref-4-1-Yang,ref-5-1-Wu,ref-6-1-贾泽,ref-7-1-伍晓明}
	
	\setlength{\biblabelsep}{12pt}
	\printbibliography[env=resumebib,heading=subbibliography,title={发表的学术论文}] % 发表的和录用的合在一起

	\end{refsection}


	\begin{refsection}[ref/resume.bib]
	\settoggle{bbx:gbtype}{false}%局部设置不输出文献类型和载体标识符
	\settoggle{bbx:gbannote}{true}%局部设置输出注释信息
	\setcounter{gbnamefmtcase}{1}%局部设置作者的格式为familyahead格式
	\nocite{ref-8-1-任天令,ref-9-1-Ren}%
	
	\setlength{\biblabelsep}{12pt}
	\printbibliography[env=resumebib,heading=subbibliography,title={研究成果}]

	\end{refsection}

\else

  \section*{发表的学术论文} % 发表的和录用的合在一起

  \begin{enumerate}[label={[\arabic*]}, leftmargin=0pt, itemindent=4em]
  \addtolength{\itemsep}{-.36\baselineskip}%缩小条目之间的间距,下面类似
  \item Yang Y, Ren T L, Zhang L T, et al. Miniature microphone with silicon-
    based ferroelectric thin films. Integrated Ferroelectrics, 2003,
    52:229-235. (SCI 收录, 检索号:758FZ.)
  \item 杨轶, 张宁欣, 任天令, 等. 硅基铁电微声学器件中薄膜残余应力的研究. 中国机
    械工程, 2005, 16(14):1289-1291. (EI 收录, 检索号:0534931 2907.)
  \item 杨轶, 张宁欣, 任天令, 等. 集成铁电器件中的关键工艺研究. 仪器仪表学报,
    2003, 24(S4):192-193. (EI 源刊.)
  \item Yang Y, Ren T L, Zhu Y P, et al. PMUTs for handwriting recognition. In
    press. (已被 Integrated Ferroelectrics 录用. SCI 源刊.)
  \item Wu X M, Yang Y, Cai J, et al. Measurements of ferroelectric MEMS
    microphones. Integrated Ferroelectrics, 2005, 69:417-429. (SCI 收录, 检索号
    :896KM.)
  \item 贾泽, 杨轶, 陈兢, 等. 用于压电和电容微麦克风的体硅腐蚀相关研究. 压电与声
    光, 2006, 28(1):117-119. (EI 收录, 检索号:06129773469.)
  \item 伍晓明, 杨轶, 张宁欣, 等. 基于MEMS技术的集成铁电硅微麦克风. 中国集成电路,
    2003, 53:59-61.
  \end{enumerate}

  \section*{研究成果} % 有就写,没有就删除
  \begin{enumerate}[label=\textbf{[\arabic*]}, leftmargin=0pt, itemindent=4em]
  \addtolength{\itemsep}{-.36\baselineskip}%
  \item 任天令, 杨轶, 朱一平, 等. 硅基铁电微声学传感器畴极化区域控制和电极连接的
    方法: 中国, CN1602118A. (中国专利公开号.)
  \item Ren T L, Yang Y, Zhu Y P, et al. Piezoelectric micro acoustic sensor
    based on ferroelectric materials: USA, No.11/215, 102. (美国发明专利申请号.)
  \end{enumerate}
\fi
\fi
\end{resume}

% 最后,需要的话还要生成附录,全文随之结束。
\appendix
\backmatter
\input{data/appendix01}

\end{document}
